\documentclass[11pt,a4paper]{article}

\usepackage{amsmath}
\usepackage{dsfont}
\usepackage{amsthm}
\usepackage{mathtools}
\usepackage{tikz}
\usepackage{physics}
\usepackage{graphicx}
\usepackage{caption}
\usepackage{subcaption}
\usepackage{listings}
\usepackage{relsize}
\usepackage[sort]{natbib}
%\usepackage[a4paper, top=1in, left=1.0in, right=1.0in, bottom=1in, includehead, includefoot]{geometry}
\setlength{\textheight}{24.0cm}
\setlength{\textwidth}{16.0cm}
\setlength{\parindent}{0.0cm}
\setlength{\topmargin}{-1.0cm}
\setlength{\oddsidemargin}{0.0cm}
\renewcommand{\baselinestretch}{1.2}
\numberwithin{equation}{section}
\theoremstyle{plain}
\newtheorem{theorem}{Theorem}[section]
\newtheorem{lemma}[theorem]{Lemma}
\newtheorem{proposition}[theorem]{Proposition}
\newtheorem{corollary}[theorem]{Corollary}
\theoremstyle{definition}
\newtheorem{definition}[theorem]{Definition}
\theoremstyle{remark}
\newtheorem{remark}[theorem]{Remark}

\DeclareMathOperator*{\argmin}{arg\,min}

\begin{document}
For the density construction there were three approaches to the covariance matrix for the density function.

\begin{enumerate}
	\item Diagonal covariance matrix with the same value in each of the diagonal entries.
	\item Diagonal covariance matrix with the first two diagonal enties proportional to the segment radius and the third entry proportional to the segment length.
	\item Not necessarily diagonal matrix made by multiplying the matrix from step 2 on the right and left by the rotation matrices $R$ and $R^{-1}$ respectively.  
\end{enumerate}	


HERE ARE THE ACKNOWLEDGEMENTS
	
\bibliographystyle{chicago}
\bibliography{master_bibliography}    

\end{document}	